\section{\label{sec:app-a} Explicit expressions for quantities entering the GPF functional}
Here we explicitly present quantities entering the GPF expressions~\eqref{Xi_L} and~\eqref{Xi_L_1}.
%\begin{equation}
%	\label{tilde_frak_M0}
%	\tilde{\frak M}_0 = \frak M_0 + (h + {\frak M_3}/{\frak M_4})\tilde{\frak M}_1 - \frac{\alpha(0)}{2}\tilde{\frak M}_1^2,
%\end{equation}

First, for the coefficients in~\eqref{Xi_L} one has
\begin{eqnarray*}
	\label{tilde_frak_M0}
	\tilde{\mathfrak M}_0 & = & \langle N \rangle_0 
	\left[
		\tilde{\mathfrak{m}}_0
		+ \left(h + \frac{\mathfrak{m}_3}{\mathfrak{m}_4} \right) \tilde{\mathfrak{m}}_1
		- \frac{\beta \hat{\Phi}_0}{2}\frac{\langle N \rangle_0}{V} \tilde{\mathfrak{m}}_1^2.
	\right],
\\
	\tilde{\mathfrak{m}}_0 & = & -\frac{\mathfrak{m}_1 \mathfrak{m}_3}{\mathfrak{m}_4} + \frac{\mathfrak{m}_2 \mathfrak{m}^2_3}{2 \mathfrak{m}_4^2} - \frac{\mathfrak{m}_3^4}{8\mathfrak{m}_4^2},
\\
	\tilde{\mathfrak{m}}_1 & = & \mathfrak{m}_1 - \frac{\mathfrak{m}_2 \mathfrak{m}_3}{\mathfrak{m}_4} + \frac{\mathfrak{m}_3^3}{3\mathfrak{m}_4^2},
\end{eqnarray*}
where $\langle N \rangle_0$ is the average particle number for the RS, and
\begin{eqnarray*}
	\mathfrak{m}_1 & = & 1,
	\\
	\mathfrak{m}_2 & = & 1 + \rho \hat{h}^{(2)},
	\\
	\mathfrak{m}_3 & = & 1 + 3\rho \hat{h}^{(2)} + \rho^2 \hat{h}^{(3)},
	\\
	\mathfrak{m}_4 & = & 1 + 7\rho \hat{h}^{(2)} + 6\rho^2 \hat{h}^{(3)} + \rho^3 \hat{h}^{(4)}.
\end{eqnarray*}
Here $\hat{h}^{(n)}$ are the Fourier transforms of the total correlation functions at $\abs{\vb k} = 0,$ and $\mathfrak{m}_n$ are the $n$-particle structure factors at $\abs{\vb k} = 0$, both determined for the RS. They are functions of the RS particle density. More detailed investigation of quantities $\mathfrak{m}_n$ and $\hat{h}^{(n)}$ was performed in~\cite{RomaJPS2024,Roma2023Preprint}.

The quantity $h$ stands for
\begin{equation*}
	h = \beta(\mu - \mu_0),
\end{equation*}
where $\mu$ and $\mu_0$ are the chemical potentials of the whole system and of the RS, respectively.

The quantity $Q(\tilde{\mathfrak{M}}_2, \mathfrak{M}_4)$ is determined by
\begin{equation*}
	\label{quantity_Q}
	Q(\tilde{\mathfrak{M}}_2, \mathfrak{M}_4) = \frac{1}{2\sqrt{\pi}} \left(\frac{12}{N_0 \langle N \rangle_0\abs{\mathfrak{m}_4}}\right)^{1/4} {\rm e}^{y^2/2} U(0,y)
\end{equation*}
where
\begin{eqnarray*}
	y & = & \left(\frac{\langle N \rangle_0}{N_0} \frac{3\tilde{\mathfrak{m}}_2^2}{\abs{\mathfrak{m}_4}}\right)^{1/2},
	\\
	\tilde{\mathfrak{m}}_2 & = & \mathfrak{m}_2 - \frac{\mathfrak{m}_3^2}{2\mathfrak{m}_4},
\end{eqnarray*}
and $U(a, y)$ is the Weber parabolic cylinder function~\cite{nistMathFuncHandbook2010}.

Now, for quantities in~\eqref{Xi_L_1} one has
\begin{eqnarray*}
	a_2 & = & \left(\frac{3}{N_0 \langle N \rangle_0 \abs{\mathfrak{m}_4}}\right)^{1/2} U(y),
	\\
	a_4 & = & \frac{3}{N_0 \langle N \rangle_0 \abs{\mathfrak{m}_4}} \varphi(y),
\end{eqnarray*}
where
\begin{eqnarray*}
	U(y) & = & \frac{U(1,y)}{U(0,y)},
	\\
	\varphi(y) & = & 3U^2(y) + 2yU(y) -2.
\end{eqnarray*}
By multiplying $a_2$ by $\langle N \rangle_0$, and $a_4$ by $\langle N \rangle_0^2$, we get quantities that are functions of the particle density only
\begin{eqnarray*}
	a_2^{'} & = & \langle N \rangle_0 a_2,
	\\
	a_4^{'} & = & \langle N \rangle_0^2 a_4.
\end{eqnarray*}

Finally, the quantity $\mu^*$ is a linear function of the chemical potential $\mu$
\begin{equation*}
	\mu^* = h + \frac{\mathfrak{m}_3}{\mathfrak{m}_4} + \frac{\langle N \rangle_0}{V} \beta\hat{\Phi}_0 \tilde{\mathfrak{m}}_1.
\end{equation*}