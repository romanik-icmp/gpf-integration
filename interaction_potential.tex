\section{\label{sec:parab_pot} Interaction potentials. Parabolic approximation}
The potential energy of the inter-particle interaction is written in the form
\begin{eqnarray*}
	U_N(\vb r^N) &=& \frac12 \underset{i\neq j}{\sum_{i=1}^N \sum_{j=1}^N} \Psi(r_{ij}) 
	+ \frac12 \underset{i\neq j}{\sum_{i=1}^N \sum_{j=1}^N} \Phi(r_{ij}).	
\end{eqnarray*}
Here $\Psi(r)$ corresponds for the short-range repulsive interaction, and $\Phi(r)$ for the long-range attractive interaction. In this work the hard-sphere (HS) potential is taken for $\Psi(r)$
\begin{equation*}
	\Psi(r) = 
	\left\{
	\begin{array}{cc}
		\infty, \quad r\leq \sigma, \\
		0, \quad r > \sigma
	\end{array}
	\right.
\end{equation*}
where $\sigma$ denotes the hard-sphere diameter. The long-range term $\Phi(r)$ is chosen so that it possesses a potential well at $r \geq \sigma$
\begin{equation}
	\label{def:long-range-potential}
	\Phi(r) = \left\{
	\begin{array}{cc}
		0, & r \leq \sigma, 
		\\
		\phi(r), & r > \sigma.
	\end{array}
	\right.
\end{equation}
where $\phi(r)$ denotes an attractive part of the interaction and is chosen in the form a few widely used potentials later.
Separation in~\eqref{def:long-range-potential} is not the only way to select the form for $\Phi(r)$ inside the hard-core region. One popular approach is the Weeks-Chandler-Andersen (WCA) regularization originated from~\cite{WCA1971}, according to which one has
\begin{equation}
	\label{def:long-range-potential_wca}
	\Phi(r) = \left\{
	\begin{array}{cc}
		-\varepsilon, & r \leq r_{\rm m}, 
		\\
		\phi(r), & r > \sigma.
	\end{array}
	\right.
\end{equation}
where $r_{\rm m}$ is the coordinate of the potential minimum. We use the WCA regularization schema for most of the potentials considered in Section~\ref{sec:results}.

It is additionally assumed that the attractive part of the interaction potential possesses a well behaved Fourier component $\hat{\Phi}_{\vb k}$ such that:
\begin{equation*}
	\Phi(r) = \frac{1}{V} \sum_{\vb k} \hat{\Phi}_{\vb k} {\rm e}^{i\vb k \vb r} = \frac{1}{(2\pi)^3} \int {\rm d} {\vb k} \hat{\Phi}_{\vb k} {\rm e}^{i\vb k \vb r},
\end{equation*}
and
\begin{equation}
	\label{def:fourier}
	\hat{\Phi}_{\vb k} = \int \Phi(r) {\rm e}^{-i\vb k \vb r} {\rm d} {\vb r}.
\end{equation}
Converting to spherical coordinates, and integrating over the angle variables, one arrives at
\begin{equation}
	\label{def:fourier_spherical}
	\hat{\Phi}(k) = \frac{4\pi}{k} \int_{0}^{\infty} r \Phi(r) \sin(k r) {\rm d} r.
\end{equation}

Several model potentials are be taken for $\phi$ in the next Section~\ref{sec:results}. 

To proceed with analytic calculations of the fixed point coordinates, and to get some numerical results, let us apply the so-called parabolic approximation for the Fourier component of the interaction potential
\begin{equation*}
	\hat{\Phi}_{\vb k} = \hat{\Phi}_0 (1 - 2b^2k^2),
\end{equation*}
where 
\begin{equation*}
	2b^2 = -\frac{1}{2\hat{\Phi}_0} \frac{\partial^2 \hat{\Phi}_k}{\partial k^2} \bigg|_{k=0}.
\end{equation*}
Then let us select the cut-off parameter as
$$
B_0 = \frac{1}{\sqrt{2} b}.
$$
For $\hat{\Phi}_{B_{n+1}, B_n}$, in the case of arithmetic average, it follows
\begin{equation*}
	\hat{\Phi}_{B_{n+1}, B_n} = \hat{\Phi}_0 \left(1 - s^{-2n} \frac{1 + s^{-2}}{2}\right)
\end{equation*}
and for $q$ one obtains
\begin{equation*}
	q = -\frac{\beta \hat{\Phi}_0}{V} \bar{q}, \quad \bar{q} = \frac{1 + s^{-2}}{2},
\end{equation*}
with $\bar{q} = 0.5389$.

In the case of spherical averaging
\begin{equation*}
	\hat{\Phi}_{B_{n+1}, B_n} = \frac{\int_{B_{n+1}}^{B_n} \hat{\Phi}_{\vb k} {\rm d} {\vb k}}
	{\int_{B_{n+1}}^{B_n} {\rm d} {\vb k}} 
	= \frac{\hat{\Phi}_0 \int_{B_{n+1}}^{B_n} (1 - 2b^2k^2) k^2 {\rm d} k}{\int_{B_{n+1}}^{B_n} k^2 {\rm d} k}
\end{equation*}
one gets
\begin{equation*}
	\hat{\Phi}_{B_{n+1}, B_n} = \hat{\Phi}_0 \left(1 - s^{-2n} \frac{3(1 - s^{-5})}{5(1 - s^{-3})}\right)
\end{equation*}
and for $q$
\begin{equation*}
	q = -\frac{\beta \hat{\Phi}_0}{V} \bar{q}, \quad \bar{q} = \frac{3(1 - s^{-5})}{5(1 - s^{-3})},
\end{equation*}
with $\bar{q} = 0.6123$ at $s = s^*$.
