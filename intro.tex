\section{Introduction}

Nowadays, computer simulations seem to be the most common tool to study the equilibrium properties of simple fluids. Still, analytical theories enabling one to calculate the thermodynamic properties of many-particle interacting systems is of invaluable importance, as they may give rise to physical understanding that otherwise would be missed. One of such theories is built around the collective variables method~\cite{Yukh1980book} with a reference system~\cite{Yukh1990}. A general overview of this approach and results obtained in its framework for liquid-gas systems near the critical point can be found in~\cite{Yukh2015En}. In this paper we focus on the details of of determining the critical temperature and how the parameters of the attractive interaction affect this temperature. 

The structure of this paper is as follows. We start with presenting the explicit functional of the grand partition function, with all coefficients explicitly defined. Then we proceed with the ``layer-by-layer'' integration of that functional to obtain a sequence of effective block Hamiltonians each characterized by its coefficients. After the result of integration over $n$ layers is written down in a generic form, we pass to the analysis of the recurrence relations between the effective Hamiltonian coefficients. As a result, we find the fixed point solution, write the recurrence relations in the linear approximation around the fixed point, and find the condition resulting into the equation for the critical temperature. In Section~\ref{sec:results} we calculate the critical temperature using the found expression for different hard-core van der Waals fluids and compare the obtained values with known results for the considered models.

