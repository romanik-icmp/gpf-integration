\section{Introduction}

Nowadays, computer simulations seem to be the most common tool to study the equilibrium properties of simple fluids. Still, analytical theories that enable the calculation of thermodynamic properties for many-particle interacting systems remain invaluable, as they may provide physical understanding that might otherwise be missed. One such theory is built around the collective variables (CV) method~\cite{Yukh1980book} with a reference system (RS)~\cite{Yukh1990}. A general overview of this approach and the results obtained in its framework for fluid systems near the liquid-vapour critical point can be found in~\cite{Yukh2015En}. For an overview of the general state of liquids physics we refer to~\cite{hansen2013theory,adamBulavin2006book}. In this paper we focus on the details of determining the critical temperature and how the parameters of the attractive interaction affect this temperature. 

The structure of this paper is as follows. In Section~\ref{sec:init-gpf}, we present a functional of the grand partition function (GPF), with all coefficients explicitly defined. Then, we proceed with the ``layer-by-layer'' integration of that functional to obtain a sequence of effective block Hamiltonians, each characterized by its own coefficients. After the result of integration over $n$ layers is written down in a generic form, we pass to the analysis of the recurrence relations between the effective Hamiltonian coefficients. As a result, we find the fixed point solution, write the recurrence relations in the linear approximation around the fixed point, and find a condition that leads to the equation for the critical temperature. In Section~\ref{sec:parab_pot}, we briefly discuss the interaction potentials and applied approximations. In Section~\ref{sec:results}, we calculate the critical temperature for different hard-core (HC) van der Waals fluids using the derived expression and compare the obtained values with known results for the considered models.

