\section{\label{sec:init-gpf} Grand partition function in the representation of collective variables}
The grand partition function (GPF) of a simple many-particle interacting system is represented as
\begin{equation}
	\label{Xi_as_prod}
	\Xi = \Xi_0\Xi_G\Xi_L.
\end{equation}
Here $\Xi_0$ is the GPF of a reference system, which is assumed to be known. $\Xi_G$ is a short-wave contribution to the GPF with wave-vectors $\abs{\vb{k}}>B$, with $B$ being the cut-off parameter.
$\Xi_L$ denotes long-wave contributions to the GPF and is the object of our investigation in this paper. The expression for $\Xi_L$, obtained in~\cite{Roma2023Preprint} is following:
\begin{equation}
	\label{Xi_L}
	\Xi_L = jQ(\tilde{\mathfrak{M}_2}, \mathfrak{M}_4)^{N_B} \exp(\tilde{\mathfrak{M}}_0). \Xi_L^{(1)}
\end{equation} 
Explicit expressions for $Q(\tilde{\mathfrak{M}_2}, \mathfrak{M}_4)$ and $\tilde{\mathfrak{M}_0}$ are given in Appendix~A, $\Xi_L^{(1)}$ is defined as
\begin{eqnarray}
	\label{Xi_L_1}
	\Xi_L^{(1)} &=& 
	\int \exp\left(
	\mu^* \rho_0 - \frac{1}{2} \sum_{\substack{\vb k \\ k \leq B}} d(k) \rho_{\vb k} \rho_{-\vb k} 
	\right.\\
	&& -\left. \frac{a_4}{4! N_B} \sum_{\substack{\vb k_1, \dotsc, \vb k_4 \\ k_i \leq B}} \rho_{\vb k_1} \dotsc \rho_{\vb k_4} \delta_{\vb{k}_1 + \dotsc + \vb{k}_4} \right) ({\rm d} \rho)^{N_B}
	\nonumber.
\end{eqnarray}
Here $d(k) = a_2 + \frac{\beta\hat{\Phi}_{\vb k}}{V}$, with $\beta$ being the inverse temperature, $V$ the volume, $\hat{\Phi}_{\vb k}$ the Fourier component of the long-range part of the interaction potential.
Quantities $\mu^*$, $a_2$ and $a_4$ are functions of the reference-system packing fraction $\eta$, temperature $T$, and microscopic parameters of interaction potential. They are explicitly presented in Appendix~A. The quantity $\mu^*$ also linearly depends on the chemical potential $\mu$.

The wave vector $\vb k$ takes on $N_B$ values in a sphere of radius $B$, so that
\begin{equation}
	\label{NB}
	N_B = \frac{B^3}{6\pi^2}V.
\end{equation}
Thus the number of variables to be integrated over is equal to $N_B$
\begin{equation*}
	(d\rho)^{N_B} = {\rm d}\rho_0 \prod_{\vb k\atop k\le B} {\rm d}\rho_{\vb k}^c {\rm d}\omega_{\vb k}^s
\end{equation*}
where $\rho_{\vb k}^c$ and $\rho_{\vb k}^s$ are the real and imaginary parts of the collective variable $\rho_{\vb k} = \rho_{\vb k}^c - {\rm i} \rho_{\vb k}^s$ respectively. The 'prime' sign over the product means that the wave-vector takes on values only form the upper semi-sphere, i.e. $k_z > 0$ and $\vb k \neq 0$.

The following simplification is used for the Kronecker symbol in~\eqref{Xi_L_1}
\begin{equation*}
	\delta_{\vb k} = \frac{1}{V} \int (\rm{d} \vb r) \rm{e}^{-{\rm i} \vb k \vb r} = \frac{1}{N_B}\sum_{\vb l} {\rm e}^{-{\rm i} \vb k \vb l}.
\end{equation*}
The summation over $\vb l$ implies that $\vb l$ takes on $N_B$ values in real space corresponding to the $N_B$ values of wave vector $\vb k$ in the sphere of radius $B$ in the reciprocal space. This is called a spherical approximation for the first Brillouin zone. It is assumed that a proper correspondance can be established between a spherical Brillouin zone and a structure in real space by analogy to how simple cubic lattice corresponds to its Brillouin zone in the Ising-model problem [REFERENCE NEEDED]. 

In what follows we also will understand that $\vb l \in \Lambda_0$ corresponds to vectors $\vb k$ such that $k \leq B$, $\vb l \in \Lambda_1$ to $k \leq B_1$, and in general $\vb l \in \Lambda_n$ to $k \leq B_n$. Here $B_n = B/s^n$ and $s$ is the renormalization parameter to be introduced later.

The expression~\eqref{Xi_L_1} formally coincides with the expression for the partition function functional of the 3-dimentional Ising-like model in an external field [REFERENCE NEEDED]. 

A necessary condition for this functional to give a critical-point solution is $\mu^* = 0$, which leads to a line of critical temperature dependence on the chemical potential $\mu$ (at some value of the reference system packing fraction $\eta_c$).

We are going to integrate~\eqref{Xi_L_1} following the method developed for calculation of the partition function of the 3-dimentional Ising-like models [Kozlovakii, CMP, 2005; Yukhnovskii, Nuovo, 1989]. The main idea is to divide the interval $[0, B]$ into sub-intervals $(B_1, B]$, $(B_2, B_1]$, $(B_3, B_2]$ and so on, where $B_1 = B/s$, $B_2 = B_1/s = B_2/s^2$, or in general $B_n = B/s^n$, $s$ being the renormalization group parameter, $s > 1.$ Variables $\rho_{\vb k}$ with $B_1 < k \leq B$ are said to belong to the first layer, $\rho_{\vb k}$ with $B_2 < k \leq B_1$ to the second one, and continuing in the same manner, $\rho_{\vb k}$ with $B_n < k \leq B_{n-1}$ to the $n-$th layer.
The integration is performed iteratively, starting with integration over the collective variables of the first layer, then over the second one and so on.
The number of variables to be left after the integration over the first layer is $N_1 = N_B / s^3 = (B^3 V)/(2\pi^2 s^3)$. Thus the number of variables integrated out in the first iteration is $N_B - N_1 = N_B(1-s^{-3})$.

To factorize the integrals the Fourier transform $\hat{\Phi}_{\vb k}$ of the long-range part of interaction potential is replaced with its average value over each interval:
\begin{eqnarray*}
	\hat{\Phi}_{\vb k} & \rightarrow & \hat{\Phi}_{B_1, B}, \quad B_1 < k \leq B;
	\\
	& & \hat{\Phi}_{B_2, B_1}, \quad B_2 < k \leq B_1;
	\\
	& & ...
	\\
	& & \hat{\Phi}_{B_n, B_{n-1}}, \quad B_n < k \leq B_{n-1}.
\end{eqnarray*}
The particulars of this averaging are not so important to outline the method of layer-by-layer integration. Thus we will return to them later when we present some numerical and graphical results.
Now let us explicitly integrate over the variables of the first layer.