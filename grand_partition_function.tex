\section{\label{sec:init-gpf} Grand partition function in the representation of collective variables}
The grand partition function (GPF) of a simple many-particle interacting system can be represented as~\cite{Yukh1990,YukhJSP1995,RomaJPS2024}
\begin{equation}
	\label{Xi_as_prod}
	\Xi = \Xi_0\Xi_G\Xi_L.
\end{equation}
Here $\Xi_0$ is the GPF of a reference system (RS), which is assumed to be known. $\Xi_G$ is a short-wave contribution to the GPF with wave-vectors $\abs{\vb{k}}>B_0$, $B_0$ being the cut-off parameter.
The quantity $\Xi_L$ denotes long-wave contributions to the GPF and is the object of our investigation in this paper. In our previous paper~\cite{RomaJPS2024} we provided very detailed derivation of the expression for $\Xi_L$ (see~\cite{Roma2023Preprint} for even more details) and presented it as follows:
\begin{equation}
	\label{Xi_L}
	\Xi_L = j_0Q(\tilde{\mathfrak{M}_2}, \mathfrak{M}_4)^{N_0} \exp(\tilde{\mathfrak{M}}_0) \Xi_L^{(1)},
\end{equation} 
where $j_0=\sqrt{2}^{N_0 - 1},$ $N_0$ being determined in~\eqref{def:NB} below, $Q(\tilde{\mathfrak{M}_2}, \mathfrak{M}_4)$ and $\tilde{\mathfrak{M}_0}$ are explicitly given in Appendix~\ref{sec:app-a}.

The quantity $\Xi_L^{(1)}$ stands for
\begin{eqnarray}
	\label{Xi_L_1}
	\Xi_L^{(1)} &=& 
	\int \exp\left(
	\mu^* \rho_0 - \frac{1}{2} \sum_{\substack{\vb k \\ k \leq B_0}} d_2(k) \rho_{\vb k} \rho_{-\vb k} 
	\right.\\
	&& -\left. \frac{a_4}{4! N_0} \sum_{\substack{\vb k_1, \dotsc, \vb k_4 \\ k_i \leq B_0}} \rho_{\vb k_1} \dotsc \rho_{\vb k_4} \delta_{\vb{k}_1 + \dotsc + \vb{k}_4} \right) ({\rm d} \rho)^{N_0}
	\nonumber.
\end{eqnarray}
Here $d_2(k) = a_2 + \frac{\beta\hat{\Phi}_{\vb k}}{V}$, $\beta$ being the inverse temperature, $V$ the volume, $\hat{\Phi}_{\vb k}$ the Fourier component of the long-range part of the interaction potential.
Quantities $\mu^*$, $a_2$ and $a_4$ are functions of the RS particle density $\rho$, temperature $T$, and microscopic parameters of interaction potential. They are explicitly presented in Appendix~\ref{sec:app-a}. The quantity $\mu^*$ also linearly depends on the chemical potential $\mu$.

The wave vector $\vb k$ takes on $N_0$ values in a sphere of radius $B_0$, so that
\begin{equation}
	\label{def:NB}
	N_0 = \frac{B_0^3}{6\pi^2}V.
\end{equation}
Thus the number of variables to be integrated over is equal to $N_0$
\begin{equation*}
	(d\rho)^{N_0} = {\rm d}\rho_0 \prod_{\vb k\atop k\le B_0}' {\rm d}\rho_{\vb k}^c {\rm d}\omega_{\vb k}^s
\end{equation*}
where $\rho_{\vb k}^c$ and $\rho_{\vb k}^s$ are the real and imaginary parts of the collective variable $\rho_{\vb k} = \rho_{\vb k}^c - {\rm i} \rho_{\vb k}^s$, respectively\footnote{Traditionally, the collective variables are denoted by $\rho_{\vb k}$, and the element of integration over CVs is denoted by $({\rm d} \rho)$, while the particle density is denoted by $\rho$. We hope that it is clear from the context when $\rho$ is understood as the number density, and when it is related to CVs}. The 'prime' sign over the product means that the wave-vector takes on values only form the upper semi-sphere, i.e. $k_z > 0$ and $\vb k \neq 0$.

The following simplification is used for the Kronecker symbol in~\eqref{Xi_L_1}
\begin{equation*}
	\delta_{\vb k} = \frac{1}{V} \int (\rm{d} \vb r) \rm{e}^{-{\rm i} \vb k \vb r} = \frac{1}{N_0}\sum_{\vb l_0} {\rm e}^{-{\rm i} \vb k \vb l_0}.
\end{equation*}
The summation over $\vb l_0$ implies that $\vb l_0$ takes on $N_0$ values in real space corresponding to the $N_0$ values of wave vector $\vb k$ in the sphere of radius $B_0$ in the reciprocal space. This is called a spherical approximation for the first Brillouin zone. It is assumed that a proper correspondence can be established between a spherical Brillouin zone and a structure in real space by analogy to how simple cubic lattice corresponds to its Brillouin zone in the Ising-model problem [REFERENCE NEEDED]. 

In what follows we also will understand that $\vb l \in \Lambda_0$ corresponds to vectors $\vb k$ such that $k \leq B_0$, $\vb l \in \Lambda_1$ to $k \leq B_1$, and in general $\vb l \in \Lambda_n$ to $k \leq B_n$. Here $B_n = B/s^n$ and $s$ is the renormalization parameter to be introduced later.

The expression~\eqref{Xi_L_1} formally coincides with the expression for the partition function functional of the 3-dimentional Ising-like model in an external field [REFERENCE NEEDED]. 

A necessary condition for this functional to give rise to a critical-point solution is $\mu^* = 0$, which leads to a line of critical temperature dependence on the chemical potential $\mu$ (at some value of the reference system particle density $\rho_c$)~\cite{RomaJPS2024}.

We are going to integrate~\eqref{Xi_L_1} following the method developed for calculation of the partition function of the 3-dimentional Ising-like models \cite{MpkCMP2005,Yukh1989riv}. The main idea is to divide the interval $[0, B_0]$ into sub-intervals $(B_1, B_0]$, $(B_2, B_1]$, $(B_3, B_2]$ and so on, where $B_1 = B_0/s$, $B_2 = B_1/s = B_2/s^2$, or in general $B_n = B_0/s^n$, $s$ being the renormalization group parameter, $s > 1.$ Variables $\rho_{\vb k}$ with $B_1 < k \leq B_0$ are said to belong to the first layer, $\rho_{\vb k}$ with $B_2 < k \leq B_1$ to the second one, and continuing in the same manner, $\rho_{\vb k}$ with $B_n < k \leq B_{n-1}$ to the $n$-th layer.
The integration is performed iteratively, starting with integration over the collective variables of the first layer, then over the second one and so on.
The number of variables to be left after the integration over the first layer is $N_1 = N_0 / s^3 = (B_0^3 V)/(2\pi^2 s^3)$. Thus the number of variables integrated out in the first iteration is $N_0 - N_1 = N_0(1-s^{-3})$.

To factorize the integrals the Fourier transform $\hat{\Phi}_{\vb k}$ of the long-range part of interaction potential is replaced with its average value over each interval:
\begin{eqnarray*}
	\hat{\Phi}_{\vb k} & \rightarrow & \hat{\Phi}_{B_1, B_0}, \quad B_1 < k \leq B_0;
	\\
	& & \hat{\Phi}_{B_2, B_1}, \quad B_2 < k \leq B_1;
	\\
	& & ...
	\\
	& & \hat{\Phi}_{B_n, B_{n-1}}, \quad B_n < k \leq B_{n-1}.
\end{eqnarray*}
The particulars of this averaging are not so important to outline the method of layer-by-layer integration. Thus we will return to them later when we present some numerical and graphical results.

\subsection{Integration over the first layer}
Now let us explicitly integrate over the variables of the first layer.

The second term in the exponent of~\eqref{Xi_L_1} is rewritten as
\begin{equation*}
	\sum_{\substack{\vb k, k \leq B_0}} d_2(k) \rho_{\vb k} \rho_{-\vb k} 
	= 
	\sum_{\substack{\vb k, k \leq B_1}} d_2(k) \rho_{\vb k} \rho_{-\vb k}
	+
	\sum_{\substack{\vb k, B_1 < k \leq B_0}} d_2(B_1, B_0) \rho_{\vb k} \rho_{-\vb k}
\end{equation*}
where $d_2(B_1, B_0) = a_2 + \beta\hat{\Phi}_{B_1, B_0}/V$.
The expression for $\Xi_L^{(1)}$ is now recast
\begin{eqnarray*}
	\Xi_L^{(1)} &=& 
	\int \exp\left(
	\mu^* \rho_0 - \frac{1}{2} \sum_{\substack{\vb k \\ k \leq B_1}} d_2(k) \rho_{\vb k} \rho_{-\vb k}
	- \frac{d_2(B_1, B_0)}{2} \sum_{\substack{\vb k \\ B_1 < k \leq B_0}} \rho_{\vb k} \rho_{-\vb k}  
	\right.
	\\
	&& -\left. \frac{a_4}{4! N_0} \sum_{\substack{\vb k_1, \dotsc, \vb k_4 \\ k_i \leq B_0}} \rho_{\vb k_1} \dotsc \rho_{\vb k_4} \delta_{\vb{k}_1 + \dotsc + \vb{k}_4} \right) 
	({\rm d} \rho)^{N_1} ({\rm d} \rho)^{N_0 - N_1}
	\nonumber.
\end{eqnarray*}
To distinguish the variables to be integrated over in the first iteration, let us denote them by $\eta_{\vb k}$, i.e. $\rho_{\vb k} \rightarrow \eta_{\vb k}$ for $B_1 < k \leq B_0$. Let us also extend the number of variables $\eta_{\vb k}$ with the help of $\delta-$functions:
\begin{equation*}
	\prod_{\vb{k}, 0\leq k \leq B_1}\delta(\eta_{\vb k}-\rho_{\vb k})
	= \int ({\rm d}\nu)^{N_1}\exp \left(2\pi{\rm i}\sum_{k\leq B_1}\nu_{\vb k}(\eta_{\vb k}-\rho_{\vb k})\right)
\end{equation*}
so that $\Xi_L^{(1)}$ is rewritten as 
\begin{eqnarray*}
	\Xi_L^{(1)} &=& 
	\int ({\rm d} \rho)^{N_1} \exp 
	\left(
		\mu^* \rho_0 - \frac{1}{2} \sum_{\substack{\vb k \\ k \leq B_1}} (d_2(k) - d_2(B_1, B_0) ) \rho_{\vb k} \rho_{-\vb k}
	\right)
	\\
	&&  \int ({\rm d}\nu)^{N_1} \exp \left(-2\pi{\rm i}\sum_{\vb k, k \leq B_1} \nu_{\vb k}\rho_{\vb k}\right)
	I(\nu_{\vb k}).
\end{eqnarray*}
Here the notation $I(\nu_{\vb k})$ stands for the integral over $\eta_{\vb k}$
\begin{eqnarray*}
	I(\nu_{\vb k}) & = & \int ({\rm d} \eta)^{N_0} \exp
	\left( 2\pi{\rm i} \sum_{\vb k, k \leq B_0} \bar{\nu}_{\vb k} \eta_{\vb k} - \frac{d_2(B_1, B_0)}{2} \sum_{\vb k, k \leq B_0} \eta_{\vb k} \eta_{-\vb k}
	\right.
	\\
	&& - \left. \frac{a_4}{4! N_0} \sum_{\substack{\vb k_1, \dotsc, \vb k_4 \\ k_i \leq B_0}} \eta_{\vb k_1} \dotsc \eta_{\vb k_4} \delta_{\vb{k}_1 + \dotsc + \vb{k}_4} \right)
\end{eqnarray*}
where the quantity $\bar{\nu}_{\vb k}$ is introduced as
\begin{equation*}
	\bar{\nu}_{\vb k} = \left\{
	\begin{array}{ll}
		\nu_{\vb k}, \quad k \leq B_1, \\
		0, \quad B_1 < k \leq B_0
	\end{array}
	\right.
\end{equation*}
by analogy with the method described in~\cite{MpkCMP2005}.

Now this integral can be factorized in the so-called site variables
\begin{equation*}
	\tilde{\eta}_{\vb l} = \frac{1}{\sqrt{N_0}}\sum_{\vb k, k \leq B_0}\eta_{\vb k}{\rm e}^{{\rm i}{\vb k}{\vb l}}
	, \qquad 
	\tilde{\bar{\nu}}_{\vb l}=\frac{1}{\sqrt{N_0}}\sum_{\vb k, k \leq B_0}\bar{\nu}_{\vb k}{\rm e}^{-{\rm i}{\vb k}{\vb l}}.
\end{equation*}
For some useful relations for site variables, see Appendix~B.
Now the integral takes on the form
\begin{eqnarray*}
	I(\nu_{\vb k}) & = & j_0^{-1} \prod_{\vb l \in \Lambda_0} \int_{-\infty}^{\infty} {\rm d} \tilde{\eta_{\vb l}}
	\exp(2\pi{\rm i} \tilde{\bar{\nu}}_{\vb l} \tilde{\eta_{\vb l}} -\frac{d_2(B_1, B_0)}{2} \tilde{\eta_{\vb l}}^2 - \frac{a_4}{4!}\tilde{\eta_{\vb l}}^4 )
	\\
	& = & j_0^{-1} [Q_{f_0}]^{N_0} \prod_{\vb l \in \Lambda_0} \exp(-\sum_{n \geq 1} \frac{S_{2n}}{(2n)!} \tilde{\bar{\nu}}_{\vb l}^{2n}).
\end{eqnarray*}
Restricting the resulting exponent to the $4$-th power in $\tilde{\bar{\nu}}_{\vb l}$ one gets
\begin{eqnarray*}
	I(\nu_{\vb k}) & = & j_0^{-1} [Q_{f_0}]^{N_0} \prod_{\vb l \in \Lambda_0}
	\exp(-\frac{S_2}{2!}\tilde{\bar{\nu}}_{\vb l}^2 - \frac{S_4}{4!}\tilde{\bar{\nu}}_{\vb l}^4 )
\end{eqnarray*}
In terms of $\nu_{\vb k}$ the result for $I(\nu_{\vb k})$ is expressed as follows:
\begin{eqnarray*}
	I(\nu_{\vb k}) & = & j_0^{-1} [Q_{f_0}]^{N_0} 
	\exp \left(- \frac{S_2}{2!} \sum_{\vb k, k \leq B_1} \nu_{\vb k} \nu_{-\vb k}
	\right.
	\\
	& & \left.
	 - \frac{S_4}{4! N_0} \sum_{\substack{\vb k_1, \dotsc, \vb k_4 \\ k_i \leq B_1}} \nu_{\vb k_1} \dotsc \nu_{\vb k_4} \delta_{\vb{k}_1 + \dotsc + \vb{k}_4}
	 \right)
	 .
\end{eqnarray*}
In the above formulae, the following quantities were introduced:
\begin{equation*}
	Q_{f_0} = \sqrt{2\pi}\left(\frac{3}{a_4}\right)^{1/4} {\rm e}^{x^2/4} U(0,x);
\end{equation*}
\begin{equation*}
	S_2 = (2\pi)^2 \left(\frac{3}{a_4}\right)^{1/2} U(x); \quad S_4 = (2\pi)^4 \frac{3}{a_4}\varphi(x);
\end{equation*}
\begin{equation*}
	x = d_2(B_1, B_0) \left(\frac{3}{a_4}\right)^{1/2}.
\end{equation*}

The next step is to integrate over $\nu_{\vb k}$ the following integral
\begin{eqnarray*}
	I_2(\rho_{\vb k}) & = & \int ({\rm d} \nu)^{N_1} 
	\exp 
	\left(-2\pi{\rm i} \sum_{\vb k, k \leq B_1} \nu_{\vb k}\rho_{\vb k} 
		-\frac{S_2}{2!} \sum_{\vb k, k \leq B_1} \nu_{\vb k}\nu_{-\vb k}
	\right.
	\\
	& &
	\left. 
		- \frac{S_4 s^{-3}}{4! N_1} \sum_{\substack{\vb k_1, \dotsc, \vb k_4 \\ k_i \leq B_1}} \nu_{\vb k_1} \dotsc \nu_{\vb k_4} \delta_{\vb{k}_1 + \dotsc + \vb{k}_4}
	\right)
	\\
	& = &
	j_1 \prod_{\vb l \in \Lambda_1} \int {\rm d} \tilde{\nu}_{\vb l} 
	\exp(-2\pi {\rm i} \tilde{\nu}_{\vb l} \tilde{\rho}_{\vb l} - \frac{S_2}{2!}\tilde{\nu}_{\vb l}^2
	- \frac{S_4}{4! s^3} \tilde{\nu}_{\vb l}^4  )
\end{eqnarray*}
where $j_1 = \sqrt{2}^{N_1 - 1}$, and this time
\begin{equation*}
	\tilde{\rho}_{\vb l} = \frac{1}{\sqrt{N_1}}\sum_{\vb k, k \leq B_1}\rho_{\vb k}{\rm e}^{{\rm i}{\vb k}{\vb l}}
	, \qquad 
	\tilde{\nu}_{\vb l}=\frac{1}{\sqrt{N_1}}\sum_{\vb k, k \leq B_1}\nu_{\vb k}{\rm e}^{-{\rm i}{\vb k}{\vb l}}.
\end{equation*}
The result of integration is
\begin{equation*}
	I_2(\rho_{\vb k}) = j_1 [Q_{\varphi_0}]^{N_1} \prod_{\vb l \in \Lambda_1} 
	\exp(- \sum_{n \geq 1} \frac{R_{2n}}{(2n)!} \tilde{\rho}_{\vb l}^{2n} ).
\end{equation*}
In the "$\rho^4$" approximation this takes the form
\begin{eqnarray*}
	I_2(\rho_{\vb k}) & = & j_1 [Q_{\varphi_0}]^{N_1} \prod_{\vb l \in \Lambda_1}
	\exp( - \frac{R_2}{2!} \tilde{\rho}_{\vb l}^2 - \frac{R_4}{4!} \tilde{\rho}_{\vb l}^4)
	\\
	& = & j_1 [Q_{\varphi_0}]^{N_1} 
	\exp( -\frac{R_2}{2!} \sum_{\vb k, k \leq B_1} \rho_{\vb k} \rho_{-\vb k} - \frac{R_4}{4!N_1} 
	\sum_{\substack{\vb k_1, \dotsc, \vb k_4 \\ k_i \leq B_1}} \rho_{\vb k_1} \dotsc \rho_{\vb k_4} \delta_{\vb{k}_1 + \dotsc + \vb{k}_4}).
\end{eqnarray*}
Here
\begin{equation*}
	Q_{\varphi_0} = (2\pi)^{-1/2} s^{3/4} \left(\frac{a_4}{\varphi(x)}\right)^{1/4} {\rm e}^{y^2/4} U(0,y);
\end{equation*}
\begin{equation*}
	R_2 = s^{3/2} \left(\frac{a_4}{\varphi(x)}\right)^{1/2} U(y); \quad R_4 = s^3 a_4 \frac{\varphi(y)}{\varphi(x)};
\end{equation*}
\begin{equation*}
	y = s^{3/2} U(x) \sqrt{\frac{3}{\varphi(x)}}.
\end{equation*}
This time the approximation for the Kronecker symbol is 
$$
	\delta_{\vb k} \approx \frac{1}{N_1} \sum_{\vb l \in \Lambda_1} \exp(-{\rm i} \vb{k} \vb{l}).
$$

Finally, as a result of integration over the first layer, we get for $\Xi_L^{(1)}$
\begin{eqnarray}
	\Xi_L^{(1)} & = & j_0^{-1}j_1 [Q_{f_0}]^{N_0} [Q_{\varphi_0}]^{N_1} 
	\\
	&& \times 
	\int ({\rm d} \rho)^{N_1} \exp
	\left(
		\mu^*\rho_0 - \frac{1}{2} \sum_{\vb k, k \leq B_1} d_2^{(1)}(k) \rho_{\vb k} \rho_{-\vb k}
	\right.
	\nonumber \\
	&& 
	\left.
		- \frac{a_4^{(1)}}{4!N_1} \sum_{\substack{\vb k_1, \dotsc, \vb k_4 \\ k_i \leq B_1}}
		\rho_{\vb k_1} \dotsc \rho_{\vb k_4} \delta_{\vb{k}_1 + \dotsc + \vb{k}_4}
	\right)
	\nonumber
\end{eqnarray}
where
\begin{equation}
	\label{RR_1_0_d}
	d_2^{(1)}(k) = a_2^{(1)} + \beta\hat{\Phi}_{\vb k} / V;
\end{equation}
\begin{equation}
	\label{RR_1_0_a2}
	a_2^{(1)} = R_2 - \frac{\beta\hat{\Phi}_{B_1, B_0}}{V} = s^{3/2} \left(\frac{a_4}{\varphi(x)}\right)^{1/2} U(y)
	- \frac{\beta\hat{\Phi}_{B_1, B_0}}{V};
\end{equation}
\begin{equation}
	\label{RR_1_0_a4}
	a_4^{(1)} = R_4 = s^3 a_4 \frac{\varphi(y)}{\varphi(x)}.
\end{equation}
These are recurrence relations between coefficients of an effective Hamiltonian before and after the integration over the first layer, establishing expressions for coefficients $a_2^{(1)}$ and $a_4^{(1)}$ via $a_2$ and $a_4$. An alternative, concise form is
\begin{eqnarray}
	\label{RR_1_0_a2_short}
	a_2^{(1)} & = & d_2(B_1, B_0) N(x) - \frac{\beta\hat{\Phi}_{B_1, B_0}}{V};
	\\
	\label{RR_1_0_a4_short}
	a_4^{(1)} & = & s^{-3} a_4 E(x).
\end{eqnarray}
Here the following quantities are introduced
\begin{equation*}
	N(x) = \frac{y U(y)}{x U(x)}; \quad E(x) = s^6 \frac{\varphi(y)}{\varphi(x)}.
\end{equation*}

\subsection{Integration over the second layer}
Following the outlined procedure, one can perform the integration over the second layer, with $B_2 < k \le B_1.$ As a result of such integration, the quantity $\Xi_L^{(1)}$ takes on the following expression:
\begin{eqnarray}
	\Xi_L^{(1)} & = & j_0^{-1}j_2 [Q_{f_0}]^{N_0} [Q_{\varphi_0}]^{N_1} [Q_{f_1}]^{N_1} [Q_{\varphi_1}]^{N_2} 
	\\
	&& \times 
	\int ({\rm d} \rho)^{N_2} \exp
	\left(
	\mu^*\rho_0 - \frac{1}{2} \sum_{\vb k, k \leq B_2} d_2^{(2)}(k) \rho_{\vb k} \rho_{-\vb k}
	\right.
	\nonumber \\
	&& 
	\left.
	- \frac{a_4^{(2)}}{4!N_2} \sum_{\substack{\vb k_1, \dotsc, \vb k_4 \\ k_i \leq B_2}}
	\rho_{\vb k_1} \dotsc \rho_{\vb k_4} \delta_{\vb{k}_1 + \dotsc + \vb{k}_4}
	\right)
	\nonumber ,
\end{eqnarray}
where $j_2 = \sqrt{2}^{N_2 - 1},$ and
\begin{equation}
	\label{RR_2_1_d}
	d_2^{(2)}(k) = a_2^{(2)} + \beta\hat{\Phi}_{\vb k} / V;
\end{equation}
\begin{equation}
	\label{RR_2_1_a2}
	a_2^{(2)} = R_2^{(1)} - \frac{\beta\hat{\Phi}_{B_2, B_1}}{V} = s^{3/2} \left(\frac{a_4^{(1)}}{\varphi(x_1)}\right)^{1/2} U(y_1)
	- \frac{\beta\hat{\Phi}_{B_2, B_1}}{V};
\end{equation}
\begin{equation}
	\label{RR_2_1_a4}
	a_4^{(2)} = R_4^{(1)} = s^3 a_4^{(1)} \frac{\varphi(y_1)}{\varphi(x_1)}.
\end{equation}
The other quantities are
\begin{equation*}
	R_2^{(1)} = s^{3/2} \left(\frac{a_4^{(1)}}{\varphi(x_1)}\right)^{1/2} U(y_1); 
	\quad 
	R_4^{(1)} = s^3 a_4^{(1)} \frac{\varphi(y_1)}{\varphi(x_1)};
\end{equation*}
\begin{equation*}
	y_1 = s^{3/2} U(x_1) \left(\frac{3}{\varphi(x_1)}\right)^{1/2};
	\quad
	x_1 = d_2^{(1)}(B_2, B_1) \left(\frac{3}{a_4^{(1)}}\right)^{1/2}.
\end{equation*}

\begin{equation*}
	Q_{f_1} = \sqrt{2\pi}\left(\frac{3}{a_4^{(1)}}\right)^{1/4} {\rm e}^{x_1^2/4} U(0,x_1);
\end{equation*}
\begin{equation*}
	Q_{\varphi_1} = (2\pi)^{-1/2} s^{3/4} \left(\frac{a_4^{(1)}}{\varphi(x_1)}\right)^{1/4} {\rm e}^{y_1^2/4} U(0,y_1).
\end{equation*}

The recurrence relations~\eqref{RR_2_1_d}~-~\eqref{RR_2_1_a4} link the coefficients of an effective Hamiltonian before and after the integration over the second layer, expressing coefficients $a_2^{(2)}$ and $a_4^{(2)}$ via $a_2^{(1)}$ and $a_4^{(1)}$. They are analogous to Eqs.~\eqref{RR_1_0_d}~-~\eqref{RR_1_0_a4}. Written in a concise form analogous to~Eqs.~\eqref{RR_1_0_a2_short} and~\eqref{RR_1_0_a4_short}, they are
\begin{eqnarray}
	a_2^{(2)} & = & d_2^{(1)}(B_2, B_1) N(x_1) - \frac{\beta\hat{\Phi}_{B_2, B_1}}{V};
	\\
	a_4^{(2)} & = & s^{-3} a_4^{(1)} E(x_1).
\end{eqnarray}

\subsection{General result for the layer-by-layer integration}
Having noticed the pattern while integration over the first two layers, we are now ready to generalize the result for an arbitrary number of layers.

\begin{eqnarray}
	\Xi_L^{(1)} & = & j_0^{-1}j_n Q_0 Q_1 \dotsc Q_{n-1} 
	\\
	&& \times 
	\int ({\rm d} \rho)^{N_n} \exp
	\left(
	\mu^*\rho_0 - \frac{1}{2} \sum_{\vb k, k \leq B_n} d_2^{(n)}(k) \rho_{\vb k} \rho_{-\vb k}
	\right.
	\nonumber \\
	&& 
	\left.
	- \frac{a_4^{(n)}}{4!N_n} \sum_{\substack{\vb k_1, \dotsc, \vb k_4 \\ k_i \leq B_n}}
	\rho_{\vb k_1} \dotsc \rho_{\vb k_4} \delta_{\vb{k}_1 + \dotsc + \vb{k}_4}
	\right)
	\nonumber ,
\end{eqnarray}
where $j_n = \sqrt{2}^{N_n - 1},$ and
\begin{equation}
	\label{RR_n_d}
	d_2^{(n)}(k) = a_2^{(n)} + \beta\hat{\Phi}_{\vb k} / V;
\end{equation}
\begin{equation}
	\label{RR_n_a2}
	a_2^{(n)} = R_2^{(n-1)} - \frac{\beta\hat{\Phi}_{B_n, B_{n-1}}}{V} = s^{3/2} \left(\frac{a_4^{(n-1)}}{\varphi(x_{n-1})}\right)^{1/2} U(y_{n-1})
	- \frac{\beta\hat{\Phi}_{B_n, B_{n-1}}}{V};
\end{equation}
\begin{equation}
	\label{RR_n_a4}
	a_4^{(n)} = R_4^{(n-1)} = s^3 a_4^{(n-1)} \frac{\varphi(y_{n-1})}{\varphi(x_{n-1})}.
\end{equation}
The other quantities are
\begin{equation*}
	R_2^{(n)} = s^{3/2} \left(\frac{a_4^{(n)}}{\varphi(x_n)}\right)^{1/2} U(y_n);
\end{equation*} 
\begin{equation*}
	R_4^{(n)} = s^3 a_4^{(n)} \frac{\varphi(y_n)}{\varphi(x_n)},
\end{equation*}
\begin{equation*}
	y_n = s^{3/2} U(x_n) \left(\frac{3}{\varphi(x_n)}\right)^{1/2},
\end{equation*}
\begin{equation*}
	x_n = d_2^{(n)}(B_{n+1}, B_n) \left(\frac{3}{a_4^{(n)}}\right)^{1/2},
\end{equation*}

\begin{equation*}
	Q_{f_n} = \sqrt{2\pi}\left(\frac{3}{a_4^{(n)}}\right)^{1/4} {\rm e}^{x_n^2/4} U(0,x_n);
\end{equation*}
\begin{equation*}
	Q_{\varphi_n} = (2\pi)^{-1/2} s^{3/4} \left(\frac{a_4^{(n)}}{\varphi(x_n)}\right)^{1/4} {\rm e}^{y_n^2/4} U(0,y_n).
\end{equation*}
The concise form of the recurrence relations is as follows:
\begin{eqnarray}
	a_2^{(n)} & = & d_2^{(n-1)}(B_n, B_{n-1}) N(x_{n-1}) - \frac{\beta\hat{\Phi}_{B_n, B_{n-1}}}{V};
	\\
	a_4^{(n)} & = & s^{-3} a_4^{(n-1)} E(x_{n-1}).
\end{eqnarray}

\subsection{Recurrence relations}
The general recurrence relations between coefficients of effective block-structure Hamiltonians have the form
\begin{eqnarray}
	a_2^{(n+1)} & = & d_2^{(n)}(B_{n+1}, B_n) N(x_n) - \frac{\beta\hat{\Phi}_{B_{n+1}, B_n}}{V};
	\\
	a_4^{(n+1)} & = & s^{-3} a_4^{(n)} E(x_{n}).
\end{eqnarray}
With the help of the following change of variables
\begin{eqnarray*}
	r_n & = & d_2^{(n)}(0)s^{2n}
	\nonumber \\
	u_n & = & a_4^{(n)}s^{4n}
	\nonumber
\end{eqnarray*}
The recurrence relations become
\begin{eqnarray}
	\label{RR_final}
	r_{n+1} & = & s^2(r_n + q) N(x_n) - s^2 q;
	\nonumber\\
	u_{n+1} & = & s u_n E(x_n).
\end{eqnarray}
Here
\begin{equation*}
	q = s^{2n} \frac{\beta \hat{\Phi}_0}{V} \left(\frac{\hat{\Phi}_{B_{n+1}, B_n}}{\hat{\Phi}_0} - 1\right) = -\frac{\beta \hat{\Phi}_0}{V} \bar{q}
\end{equation*}
where
\begin{equation}
	\bar{q} = -s^{2n} \left(\frac{\hat{\Phi}_{B_{n+1}, B_n}}{\hat{\Phi}_0} - 1\right).
\end{equation}
The recurrence relations~\eqref{RR_final} possess a fixed point solution in the limit of $n \to \infty$. By definition, the fixed point solution means
\begin{equation*}
	r_{n+1} = r_n = r^*; \quad u_{n+1} = u_n = u^*
\end{equation*}
and hence
\begin{eqnarray*}
	r^* & = & s^2[-q + (r^* + q)N(x^*)],
	\nonumber \\
	u^* & = & s u^* E(x^*).
\end{eqnarray*}
From the last equality it follows
\begin{equation}
	sE(x^*) = 1.
\end{equation}
It is practical to chose $x^* = 0$, which is equivalent to $\lim_{n \to \infty}d_2^{(n)}(0) = d_2^* = 0$ at the fixed point, and gives $s = s^* = 3.5862$. We will use this value of $s$ to obtain particular numerical and graphical results. From the first recurrence relation it follows
\begin{equation*}
	r^* = -q\frac{N(x^*) - 1}{N(x^*) - s^{-2}}.
\end{equation*}
At $x^* = 0$, $r^* = -q$ because 
$$
\lim_{x \to 0} \frac{N(x) -1}{N(x) - s^{-2}} = 1.
$$ 

Finally, from 
\begin{equation*}
	x^* = \sqrt{3} \frac{r^* + q}{\sqrt{u^*}}
\end{equation*}
one finds the following expression for $u^*$
\begin{equation*}
	u^* = \frac{3 q^2}{(x^*)^2} \left[\frac{1 - s^{-2}}{N(x^*) - s^{-2}}\right]^2.
\end{equation*}

Taking into account the results of the following Subsection~\ref{sec:parab_pot}, we can write down the coordinates of the fixed point:
\begin{equation}
	r^* = - {\frac{\beta \abs{\hat{\Phi}_0}}{V}} \bar{r}, 
	\quad 
	u^* = \left(\frac{\beta \hat{\Phi}_0}{V}\right)^2 \bar{u}
\end{equation}
where
\begin{equation*}
	\bar{r} = \frac{N(x^*) - 1}{N(x^*) - s^{-2}} \bar{q}, 
	\quad 
	\bar{u} = \frac{3}{(x^*)^2} \left(\frac{1 - s^{-2}}{N(x^*) - s^{-2}}\right)^2 \bar{q}^2.
\end{equation*}
We consider potentials with $\hat{\Phi}_0 < 0$, thus we can write $\hat{\Phi}_0 = -\abs{\hat{\Phi}_0}$

Let's use the linear approximation for the recurrence relations~\ref{RR_final}
\begin{equation}
	\label{eq:linear_rr}
	\left(
		\begin{array}{c}
			r_{n+1} - r^* \\
			u_{n+1} - u^*
		\end{array}
	\right)
	= R
	\left(
		\begin{array}{c}
			r_{n} - r^* \\
			u_{n} - u^*
		\end{array}
	\right)
\end{equation}
Let us calculate the elements of the linearized renormalization group transformation matrix $R$.
\begin{equation*}
	R_{11} = \left(\frac{\partial r_{n+1}}{\partial r_n}\right)^* = s^2 \left[N(x^*) + x^* \left(\frac{\partial N(x)}{\partial x} \right)_{x=x^*} \right],
\end{equation*}
\begin{equation*}
	R_{12} = \left(\frac{\partial r_{n+1}}{\partial u_n}\right)^* = \frac{R_{12}^{(0)}}{\sqrt{u^*}},
\end{equation*}
\begin{equation*}
	R_{21} = \left(\frac{\partial u_{n+1}}{\partial r_n}\right)^* = R_{21}^{(0)} \sqrt{u^*},
\end{equation*}
\begin{equation*}
	R_{22} = \left(\frac{\partial u_{n+1}}{\partial u_n}\right)^* = s \left[E(x^*) - \frac{x^*}{2} \left(\frac{\partial E(x)}{\partial x}\right)_{x=x^*}\right].
\end{equation*}
Here the star refers to the fixed-point values, and
\begin{equation*}
	R_{12}^{(0)} = -\frac{s^2}{2\sqrt{3}} (x^*)^2 \left(\frac{\partial N(x)}{\partial x} \right)_{x=x^*},
\end{equation*}
\begin{equation*}
	R_{21}^{(0)} = s\sqrt{3} \left(\frac{\partial E(x)}{\partial x}\right)_{x=x^*}.
\end{equation*}
Note that that following results were used for calculation the matrix elements:
\begin{equation*}
	\left(\frac{\partial x_n}{\partial r_n}\right)^* = \frac{\sqrt{3}}{\sqrt{u^*}}, 
	\quad
	\left(\frac{\partial x_n}{\partial u_n}\right)^* = - \frac{x^*}{2u^*}.
\end{equation*}

The eigenvalues for the matrix $R$ are
\begin{equation*}
	E_1 = \frac{1}{2} \left\{ (R_{11} + R_{22}) + \left[(R_{11} - R_{22})^2 + 4 R_{12}^{(0)}R_{21}^{(0)}\right]^{1/2}\right\},
\end{equation*}
\begin{equation*}
	E_2 = \frac{1}{2} \left\{ (R_{11} + R_{22}) - \left[(R_{11} - R_{22})^2 + 4 R_{12}^{(0)}R_{21}^{(0)}\right]^{1/2}\right\},
\end{equation*}
and the eigenvectors are
\begin{equation*}
	W_1 = W_{11} \left(
	\begin{array}{ll}
		1 
		\\ 
		R_1
	\end{array}
	\right)
	, \quad
	W_1 = W_{11} \left(
	\begin{array}{ll}
		R 
		\\ 
		1
	\end{array}
	\right)
\end{equation*}
where 
\begin{equation*}
	R = \frac{R_{12}}{E_2 - R_{11}}, \quad R_1 = \frac{E_1 - R_{11}}{R_{12}},
\end{equation*}
and constants $W_{11}$ and $W_{22}$ are yet to be determined.
Thus the solution to the linear recurrence relations~\eqref{eq:linear_rr} can be written as
\begin{equation}
	\left(
	\begin{array}{ll}
		r_n - r^*
		\\
		u_n - u^*
	\end{array}
	\right)
	= c'_1 W_1 E_1^n + c'_2 W_2 E_2^n,
\end{equation}
where $c'_1$ and $c'_2$ are some constants. We arrive at the following equatinos for $r_n$ and $u_n$:
\begin{eqnarray}
	r_n & = & r^* + c_1 E_1^n + c_2 R E_2^n,
	\nonumber \\
	u_n & = & u^* + c_1 R_1 E_1^n + c_2 E_2^n,
\end{eqnarray}
where $c_1 = W_{11} c'_1,$ $c_2 = W_{22} c'_2$, and are to be determined from the initial conditions at $n=0$:
\begin{equation*}
	r_0 = a_2 + \frac{\beta\hat{\Phi}_0}{V},
	\quad
	u_0 = a_4,
\end{equation*}
that leads to
\begin{eqnarray*}
	c_1 & = & (r_0 - r^* -(u_0 - u^*)R)D^{-1},
	\\
	c_2 & = &  (u_0 - u^* - (r_0 - r^*)R_1)D^{-1},
\end{eqnarray*}
where
\begin{equation*}
	D = 1 - R_1 R = \frac{E_1 - E_2}{R_{11} - E_2}.
\end{equation*}

The fixed point solution should obey the recurrence relations in the critical point. Since $E_1 > 1$, it follows that the following condition must be true
\begin{equation*}
	c_1(T_c) = 0.
\end{equation*}
In explicit form, $c_1$ is written as
\begin{equation*}
	c_1 = 
	\left(
		a_2 - (1 - \bar{r} - R^{(0)}\sqrt{\bar{u}})\frac{\beta \abs{\hat{\Phi}_0}}{V} + \frac{a_4 R^{(0)}}{\sqrt{\bar{u}}} \left(\frac{\beta \abs{\hat{\Phi}_0}}{V}\right)^{-1}
	\right) D^{-1}
\end{equation*}
and one gets the equation for the critical temperature
\begin{equation}
	\left(1 - \bar{r} - R^{(0)}\sqrt{\bar{u}} \right) \left(\frac{\beta \abs{\hat{\Phi}_0}}{V}\right)^{2} - a_2  \frac{\beta \abs{\hat{\Phi}_0}}{V} + \frac{a_4 R^{(0)}}{\sqrt{\bar{u}}} = 0,
\end{equation}
which is a quadratic equation for $\beta \abs{\hat{\Phi}_0}/{V}$. Out of two solutions to this equation, we select the one that gives positive value for the critical temperature:
\begin{equation}
	\label{eq:T_c}
	T^*_c \equiv \frac{k_{\rm B} T_c}{\varepsilon} = \frac{\abs{\hat{\Phi}_0}}{\varepsilon V}
	\left(
		\frac
		{a_2 + \sqrt{a_2^2 - 4 a_4 R^{(0)}\bar{u}^{-1/2} (1 - \bar{r} - R^{(0)}\bar{u}^{1/2}) } }
		{2(1 - \bar{r} - R^{(0)}\bar{u}^{1/2})}
	\right)^{-1}.
\end{equation}

We have obtained an explicit expression for the critical temperature. It is expressed in terms of the Fourier transform of the long-range part of interaction potential at the long-wave limit, i.e. at $\vb k = 0,$ of coefficients $a_2$ and $a_4$, which are calculated only based on the reference system, and on the details of averaging the potentials along the layer-by-layer integration (which is equivalent to the renormalization group transformation due to Wilson and Kadanoff [REFERENCE NEEDED]). 

Note also that the value of the critical temperature depends on the density, or rather packing fraction $\eta$ of the reference system. One possible approach to find the critical value of $\eta_c$ is from the condition that the average number of particles of the reference system is equal to that of the whole system $\langle N \rangle_{RS} = \langle N \rangle$, see~\cite{RomaJPS2024} for details. This condition is essentially a mean field approximation for the critical density~\cite{CaillolPatsahan2005,CaillolPatsahan2006}, but since the dependence of $T^*_c$ on $\eta$ is smooth, see Figure~2, we will use the critical value of $\eta_c = 0.13044$ found from this condition in~\cite{YukhJSP1995}.

In what follows we're going to calculate numerical values for $T^*_c$ following from Eq~\eqref{eq:T_c} for a few model system of type ``hard spheres with long-range interaction'', and compare the results obtained from our analytical approach with known results from computer simulations.
